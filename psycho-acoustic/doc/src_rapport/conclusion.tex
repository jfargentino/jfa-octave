Sur un plan technique, ce stage a confirm{\'e} mon savoir faire dans
certains domaines comme le traitement du signal ou encore la
programmation. Mais plus important, il a d{\'e}montr{\'e} que je pouvais
{\^e}tre autonome et m'adapter. En effet, Le sujet initial du stage,
"l'{\'e}tude de la distorsion photo-sonique", n'a pu {\^e}tre men{\'e} {\`a} son
terme {\`a} cause de l'absence de signaux de tests et d'une
faisabilit{\'e} incertaine. En fin de stage, j'ai {\'e}galement abord{\'e} un
troisi{\`e}me sujet, l'encapsulation sous forme d'une biblioth{\`e}que de
lien dynamique (DLL Windows) de l'algorithme de transposition
temporelle se trouvant au coeur de l'Harmo (voir pr{\'e}sentation de
l'entreprise page \pageref{genesis}).\\

De plus, le sujet de l'{\'e}tude des sons stationnaires s'{\'e}tant
concr{\'e}tis{\'e} par le d{\'e}veloppement d'une bo{\^\i}te {\`a} outils enti{\`e}rement
configurable par l'utilisateur, une partie importante de ce projet
a servi {\`a} la r{\'e}daction du manuel d'utilisation (page
\pageref{manuel}). Il est certain que l'absence d'un tel manuel
aurait grandement diminu{\'e} l'int{\'e}r{\^e}t d'une telle bo{\^\i}te d'outils,
puisqu'alors elle aurait {\'e}t{\'e} difficilement utilisable par
quelqu'un d'autre que moi.\\

Enfin, ce stage dans une petite entreprise m'a fait prendre
conscience des difficult{\'e}s qu'elles peuvent rencontrer. Car m{\^e}me
si les comp{\'e}tences de la soci{\'e}t{\'e} GENESIS dans l'{\'e}tude et le
traitement des signaux sonores ne sont plus {\`a} d{\'e}montrer, cette
derni{\`e}re a d{\'e}pos{\'e} le bilan. J'ai donc assist{\'e} aux processus
administratifs qui accompagnent cette situation, me permettant
d'apprendre quelques petites choses du droit du travail et de la
vie juridique des entreprises.\\

J'ai donc v{\'e}cu ce stage comme une v{\'e}ritable transition de l'{\'e}cole
vers le monde professionnel, et je pense {\^e}tre enfin pr{\^e}t {\`a}
travailler en entreprise.\\
