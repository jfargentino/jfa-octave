Pour cl{\^o}turer mes {\'e}tudes d'ing{\'e}nieur, j'ai effectu{\'e} un stage d'une
dur{\'e}e de 5 mois au sein de la soci{\'e}t{\'e} GENESIS. Pendant cette
p{\'e}riode, un projet m'a {\'e}t{\'e} confi{\'e}. Ce rapport pr{\'e}sente donc le
travail que j'ai fait, ainsi que les fondements th{\'e}oriques sur
lesquels il se base.

L'une des sp{\'e}cialit{\'e}s cette soci{\'e}t{\'e} est la simulation
d'environnements sonores. Pour r{\'e}pondre {\`a} ce genre de probl{\`e}me,
deux approches sont envisageables:\\

\begin{itemize}
    \item L'enregistrement des signaux sonores g{\'e}n{\'e}r{\'e}s par le syst{\`e}me
    pour l'ensemble des {\'e}tats que l'on souhaite simuler. Puis la
    relecture des {\'e}chantillons au moment de la simulation.

    Cette technique ne n{\'e}cessite pas ou peu d'analyse, mais
    elle poss{\`e}de deux inconv{\'e}nients majeurs. Le premier est que
    la quantit{\'e} de m{\'e}moire n{\'e}cessaire pour stocker les enregistrements est
    trop importante au vue de la qualit{\'e} des sons restitu{\'e}s. De
    plus, un simulateur d'environnement sonore bas{\'e} sur ce
    principe ne pourra {\'e}voluer qu'au prix du r{\'e}-enregistrement des
    signaux {\`a} lire.\\

    \item La synth{\`e}se en temps r{\'e}el des signaux sonores par
    algorithme pour laquelle une analyse du syst{\`e}me {\`a} simuler est
    n{\'e}cessaire. Dans ce cas, ce n'est plus la quantit{\'e} de m{\'e}moire
    qui est critique mais la puissance de calcul du g{\'e}n{\'e}rateur.
    En effet, l'architecture de ce dernier est conditionn{\'e}e par la
    mod{\'e}lisation choisie.\\
\end{itemize}

Pour cette derni{\`e}re m{\'e}thode, il existe deux types d'algorithmes:\\

\begin{itemize}
    \item Les algorithmes de mod{\'e}lisation physique o{\`u} on tachera de
    d{\'e}crire le syst{\`e}me {\`a} simuler par des {\'e}quations m{\'e}caniques
    rendant compte du son qu'il produit. Il faudra alors que le
    simulateur r{\'e}solve en temps r{\'e}el ces {\'e}quations pour pouvoir
    synth{\'e}tiser le son demand{\'e}. Ce genre de syst{\`e}me d'{\'e}quations
    devient tr{\`e}s vite trop complexe pour des syst{\`e}mes {\`a} simuler un
    peu {\'e}volu{\'e}s, et la qualit{\'e} du rendu sonore n'est pas toujours {\`a}
    la hauteur.\\

    \item A l'inverse, les algorithmes de mod{\'e}lisation du signal
    sonore ne d{\'e}crivent pas la source du son, mais bien le son
    lui-m{\^e}me. Il faut donc trouver un mod{\`e}le math{\'e}matique d{\'e}crivant
    plus ou moins fid{\`e}lement le signal sonore, selon la qualit{\'e}
    exig{\'e}e.\\
\end{itemize}

Il est possible d'adopter en plus une approche dite perceptive. En
effet, les sons {\`a} simuler sont de natures tr{\`e}s diverses, il est
donc impossible d'en sortir un mod{\`e}le math{\'e}matique g{\'e}n{\'e}rique. En
revanche, le r{\'e}cepteur est toujours identique puisqu'il s'agit de
l'oreille humaine. La psychoacoustique (voir chapitre 2 page
\pageref{theorie}) est la science qui analyse la perception des
sons par l'Homme, elle nous permet alors de s{\'e}lectionner dans un
signal ce qui sera vraiment per\c{c}u par l'auditeur. Des m{\'e}thodes de
compression audio avec perte (la norme MP3 par exemple) sont
bas{\'e}es sur de tels principes.

Le mod{\`e}le choisi par les ing{\'e}nieurs de la soci{\'e}t{\'e} GENESIS est une
d{\'e}composition du signal sonore en un nombre fini de sinuso{\"\i}des et
de bandes de bruit {\`a} spectre contigu{\"e}. C'est un mod{\`e}le perceptif,
c'est {\`a} dire que ces param{\`e}tres seront d{\'e}termin{\'e}s selon des
principes psychoacoustiques. Ainsi les sons d'origine, {\`a} spectres
en g{\'e}n{\'e}ral tr{\`e}s complexes, seront simplifi{\'e}s pour ne garder que
les composantes pertinentes {\`a} l'oreille. Le nombre de composantes
{\`a} garder d{\'e}pendant essentiellement de la qualit{\'e} exig{\'e}e, de la
puissance de calcul du g{\'e}n{\'e}rateur de son mais {\'e}galement de
l'efficacit{\'e} de la m{\'e}thode de s{\'e}lection.

Jusqu'{\`a} pr{\'e}sent, les ing{\'e}nieurs de GENESIS s{\'e}lectionnaient "{\`a}
l'oreille" les param{\`e}tres {\`a} utiliser lors de la synth{\`e}se, et cela
au prix d'un temps pr{\'e}cieux. Cependant, un traitement automatique
des composantes pertinentes d'un son est envisageable en
programmant certains des concepts de la psychoacoustique. J'ai
donc d{\'e}velopp{\'e} au cours de mon stage un ensemble de fonctions pour
le logiciel \matlabv permettant d'aider {\`a} l'analyse de sons
stationnaires, ainsi qu'{\`a} la s{\'e}lection de leurs param{\`e}tres
pertinents selon le mod{\`e}le retenu (sinus + bruit). L'utilisation
de cette bo{\^\i}te {\`a} outils se veut la plus souple possible, en effet
l'utilisateur a acc{\`e}s {\`a} l'ensemble des param{\`e}tres internes qu'il
pourra alors adapter {\`a} ses besoins.

Apr{\`e}s une pr{\'e}sentation de la soci{\'e}t{\'e} GENESIS (page
\pageref{genesis}), je d{\'e}velopperai succinctement les concepts de
psychoacoustique sur lesquels s'appuient les fonctions {\'e}crites
(page \pageref{theorie}). Enfin, le dernier chapitre expliquera en
d{\'e}tails le fonctionnement de chacune d'elles (page
\pageref{manuel}). De plus, un CD-ROM est fourni. Il contient la
bo{\^\i}te {\`a} outils, un certain nombre de sons pour pouvoir la tester,
le pr{\'e}sent rapport ainsi que ses sources et le support de la
pr{\'e}sentation orale.
